\documentclass{article}

% if you need to pass options to natbib, use, e.g.:
%     \PassOptionsToPackage{numbers, compress}{natbib}
% before loading neurips_2025

% The authors should use one of these tracks.
% Before accepting by the NeurIPS conference, select one of the options below.
% 0. "default" for submission
\usepackage[position, final]{neurips_2025}
\usepackage{ctex}
\usepackage{natbib}
\setcitestyle{numbers,square}
% the "default" option is equal to the "main" option, which is used for the Main Track with double-blind reviewing.
% 1. "main" option is used for the Main Track
%  \usepackage[main]{neurips_2025}
% 2. "position" option is used for the Position Paper Track
%  \usepackage[position]{neurips_2025}
% 3. "dandb" option is used for the Datasets & Benchmarks Track
 % \usepackage[dandb]{neurips_2025}
% 4. "creativeai" option is used for the Creative AI Track
%  \usepackage[creativeai]{neurips_2025}
% 5. "sglblindworkshop" option is used for the Workshop with single-blind reviewing
 % \usepackage[sglblindworkshop]{neurips_2025}
% 6. "dblblindworkshop" option is used for the Workshop with double-blind reviewing
%  \usepackage[dblblindworkshop]{neurips_2025}

% After being accepted, the authors should add "final" behind the track to compile a camera-ready version.
% 1. Main Track
 % \usepackage[main, final]{neurips_2025}
% 2. Position Paper Track
%  \usepackage[position, final]{neurips_2025}
% 3. Datasets & Benchmarks Track
 % \usepackage[dandb, final]{neurips_2025}
% 4. Creative AI Track
%  \usepackage[creativeai, final]{neurips_2025}
% 5. Workshop with single-blind reviewing
%  \usepackage[sglblindworkshop, final]{neurips_2025}
% 6. Workshop with double-blind reviewing
%  \usepackage[dblblindworkshop, final]{neurips_2025}
% Note. For the workshop paper template, both \title{} and \workshoptitle{} are required, with the former indicating the paper title shown in the title and the latter indicating the workshop title displayed in the footnote.
% For workshops (5., 6.), the authors should add the name of the workshop, "\workshoptitle" command is used to set the workshop title.
% \workshoptitle{WORKSHOP TITLE}

% "preprint" option is used for arXiv or other preprint submissions
 % \usepackage[preprint]{neurips_2025}

% to avoid loading the natbib package, add option nonatbib:
%    \usepackage[nonatbib]{neurips_2025}

\usepackage[utf8]{inputenc} % allow utf-8 input
\usepackage[T1]{fontenc}    % use 8-bit T1 fonts
\usepackage{hyperref}       % hyperlinks
\usepackage{url}            % simple URL typesetting
\usepackage{booktabs}       % professional-quality tables
\usepackage{amsfonts}       % blackboard math symbols
\usepackage{nicefrac}       % compact symbols for 1/2, etc.
\usepackage{microtype}      % microtypography
\usepackage{xcolor}         % colors

% Note. For the workshop paper template, both \title{} and \workshoptitle{} are required, with the former indicating the paper title shown in the title and the latter indicating the workshop title displayed in the footnote. 
\title{自然语言处理模型的发展现状和前景}

% The \author macro works with any number of authors. There are two commands
% used to separate the names and addresses of multiple authors: \And and \AND.
%
% Using \And between authors leaves it to LaTeX to determine where to break the
% lines. Using \AND forces a line break at that point. So, if LaTeX puts 3 of 4
% authors names on the first line, and the last on the second line, try using
% \AND instead of \And before the third author name.


\author{%
    何家睿 \\
    BEng (AI) \\
    香港科技大学(广州) \\
    广东,中国 \\
    \texttt{jhe218@connect.hkust-gz.edu.cn} \\
}


\begin{document}

\section{大纲}
\subsection{引言}
本节介绍自然语言处理的重要性与背景,阐述综述的动机与目标,并说明本文的范围与结构安排。

\subsection{技术演进与体系结构}
本节回顾自然语言处理模型从统计方法到深度学习的发展历程,重点介绍 CNN, RNN 到 Transformer 的架构演进,以及预训练—微调范式的形成与影响。

\subsection{现状:能力、应用与生态}
本节总结当前 NLP 模型在理解、生成、推理等方面的能力特点,概述其在翻译、问答、摘要等任务中的应用,并讨论开源生态、工具链与应用层现状。

\subsection{评估与基准问题}
本节介绍现有评测基准及其局限,讨论对抗性评测、事实性验证、多模态评估等新需求。

\subsection{主要挑战与限制}
本节从数据、计算成本、可靠性、可解释性、伦理与安全等方面分析当前 NLP 技术面临的关键问题。

\subsection{未来方向与潜力}
本节探讨未来的发展趋势,包括模型高效化、多模态融合、知识增强、可解释性、安全控制及个性化适配等方向。

\subsection{结论}
本节对全文进行总结,概括自然语言处理模型的主要进展、现存挑战,并对未来发展做出简要展望。

\newpage

\maketitle
\setcounter{section}{0}
\begin{abstract}
随着深度学习技术的快速发展,自然语言处理(NLP)在理解与生成语言的能力上取得了前所未有的突破。从早期依赖特征工程的统计模型,到基于深度神经网络的表示学习,再到以 Transformer 为核心的大规模预训练模型,NLP 的研究范式在过去十年间发生了显著转变。当前的大语言模型在翻译、问答、摘要、对话等任务中展现出强大的泛化能力,并推动了应用生态的快速扩张。然而,模型规模的不断增长也带来了计算成本高、数据偏差难以控制、输出可靠性有限等新的挑战。本文在系统回顾 NLP 模型技术演进的基础上,总结其当前能力、典型应用与生态特征,讨论现有评估体系及其局限,分析面临的关键问题,并进一步展望未来的发展方向,以期为相关研究与应用提供参考。
\end{abstract}

\section{引言}
自然语言处理(Natural Language Processing, NLP)\cite{liddy2001natural} 是人工智能领域最具活力和影响力的研究方向之一,其目标是使机器能够理解、生成并有效使用自然语言。随着互联网内容规模的爆炸式增长以及智能交互需求的不断提高,NLP 在搜索引擎、智能问答、机器翻译、文本分析与对话系统等应用场景中发挥着愈发重要的作用。尤其是近年来大规模预训练语言模型的兴起,使得模型能够在少量甚至零样本条件下完成复杂语言任务,进一步提升了 NLP 技术的实用性与影响力。

尽管取得了显著进展,NLP 的研究仍处于快速演化之中。从基于规则和统计的早期方法,到 RNN、CNN 等深度神经网络的兴起,再到 Transformer \cite{vaswani2017attention} 体系结构的广泛成功,NLP 的技术基础不断演进;同时,新的应用需求、评估方式与工程生态也在持续成形。然而,随着模型规模与应用范围的扩大,训练与推理成本、数据与模型偏差、可解释性与安全性等问题愈加凸显,亟需系统化的分析与讨论。

基于上述背景,本文旨在对自然语言处理模型的发展现状与未来潜力进行综述。本文的综述范围主要覆盖语言模型的体系结构演进、典型任务能力、应用生态、评估方法,以及当前面临的主要挑战。全文结构如下:第二节回顾 NLP 技术的发展路径,从统计方法到深度学习与预训练模型;第三节总结当前模型的能力与应用生态;第四节介绍常用评估基准及其局限;第五节分析主要挑战,包括数据质量、成本、可靠性与伦理问题;第六节讨论未来的潜在发展方向;第七节对全文进行总结。

\bibliographystyle{IEEEtran}
\bibliography{draft}

\end{document}