\documentclass[10pt, a4paper]{article}
\usepackage{xcolor}
\usepackage{geometry}
\usepackage{fontspec}
\usepackage{xeCJK}
\usepackage{array}
\usepackage{parskip}
\usepackage{enumitem}
\usepackage{titlesec}
\usepackage{lastpage}
\usepackage{fancyhdr}

% 设置页面边距
\geometry{left=1.5cm, right=1.5cm, top=1.5cm, bottom=1.5cm}

% 设置中文字体
\setCJKmainfont{SimSun}
\setCJKsansfont{SimHei}

% 自定义颜色
\definecolor{darkblue}{RGB}{0, 51, 102}
\definecolor{gray}{RGB}{128, 128, 128}

% 设置章节格式
\titleformat{\section}{\Large\bfseries\color{darkblue}}{}{0em}{}[\titlerule]

% 设置列表格式
\setlist[itemize]{leftmargin=*, nosep, after=\vspace{0.5ex}}
\setlist[enumerate]{leftmargin=*, nosep, after=\vspace{0.5ex}}

% 页眉页脚
\pagestyle{fancy}
\fancyhf{}
\renewcommand{\headrulewidth}{0pt}
\rfoot{第\thepage 页,共\pageref{LastPage}页}

\begin{document}

% 个人信息部分
\begin{center}
    {\Huge\bfseries\color{darkblue} 张三}\\[4pt]
    \textcolor{gray}{香港科技大学(广州) | 人工智能专业本科生 | 预计2027年毕业}\\
    \vspace{8pt}
    \begin{tabular}{m{0.45\textwidth} m{0.45\textwidth}}
        \textbf{电话}: (+86) 138-0000-0000 & \textbf{邮箱}: zhangsan@email.com \\
        \textbf{地址}: 广州市番禺区香港科技大学(广州) & \textbf{政治面貌}: 共青团员 \\
    \end{tabular}
\end{center}

\vspace{10pt}

% 求职意向
\section*{求职意向}
香港科技大学(广州)人工智能/数据科学/计算机科学相关硕士项目,期望在2027年秋季入学。

\vspace{-8pt}

% 教育背景
\section*{教育背景}

\textbf{香港科技大学(广州)} \hfill \textbf{广州}\\
\textit{工学学士(人工智能)} \hfill \textbf{2023.09 - 2027.06(预计)}\\[2pt]
\begin{itemize}
    \item \textbf{平均绩点(GPA)}: 3.7/4.0(专业前15\%)
    \item \textbf{主修课程}: 机器学习(A)、深度学习(A-)、计算机视觉、自然语言处理、算法设计与分析、数据结构、操作系统、分布式系统
    \item \textbf{学术荣誉}: 校级一等奖学金(2023-2024学年)
\end{itemize}

\vspace{-8pt}

% 竞赛经历
\section*{竞赛经历}

\textbf{ICPC国际大学生程序设计竞赛(国际大学生程序设计竞赛)}\\
\begin{itemize}
    \item \textbf{重庆赛区} \hfill \textbf{2024.11}
    \begin{itemize}
        \item 获得\textbf{银牌},团队排名前10\%
        \item 主要负责图论和动态规划类题目,解决复杂算法问题
    \end{itemize}
    
    \item \textbf{济南赛区} \hfill \textbf{2024.10}
    \begin{itemize}
        \item 获得\textbf{银牌},团队排名前15\%
        \item 担任团队算法核心,负责设计高效解题策略
    \end{itemize}
    
    \item \textbf{总训练时长超过600小时},熟练掌握各类算法和数据结构
\end{itemize}

\vspace{-8pt}

% 科研经历
\section*{科研经历}

\textbf{LLM Agent安全性与对抗攻击研究} \hfill \textbf{2025.03 - 2025.11}\\
\textit{第三作者 | 香港科技大学(广州)人工智能实验室}\\[2pt]
\begin{itemize}
    \item 研究大型语言模型(LLM)Agent的安全漏洞,设计对抗性攻击方法测试模型鲁棒性
    \item 开发自动化测试框架,识别并分类5类常见Agent安全漏洞
    \item 论文已投稿至国际会议(在审),负责实验设计与部分撰写工作
\end{itemize}

\vspace{-8pt}

% 课程设计/项目经历
\section*{课程设计与项目经历}

\textbf{基于LLM的粤语歌词生成与协音系统} \hfill \textbf{2025.09 - 2025.12}\\
\textit{自然语言处理课程设计项目 | 组长}\\[2pt]
\begin{itemize}
    \item 设计并训练LLM Agent,根据旋律音符自动填写符合粤语协音规则的歌词
    \item 使用Transformer架构,结合音乐特征编码和语言学约束,生成准确率提升25\%
    \item 实现前端交互界面,支持实时旋律输入与歌词生成
\end{itemize}

\textbf{简易操作系统内核设计} \hfill \textbf{2025.02 - 2025.06}\\
\textit{操作系统课程设计项目 | 核心开发者}\\[2pt]
\begin{itemize}
    \item 从零开始设计实现一个具有多任务调度、内存管理和文件系统的简易操作系统
    \item 使用C语言编写内核代码,实现基于时间片轮转的进程调度算法
    \item 开发系统调用接口和简单的命令行解释器,支持基本文件操作
\end{itemize}

\vspace{-8pt}

% 技能专长
\section*{技能与专长}

\begin{itemize}
    \item \textbf{编程语言}: 
    \begin{itemize}
        \item \textbf{Python}: 熟练使用PyTorch、TensorFlow、NumPy、Pandas等库,具备完整机器学习项目开发经验
        \item \textbf{C/C++}: 熟练掌握,具备操作系统开发和算法竞赛经验,熟悉STL和内存管理
    \end{itemize}
    
    \item \textbf{技术框架与工具}: 
    \begin{itemize}
        \item \textbf{机器学习}: PyTorch、TensorFlow、Scikit-learn、Hugging Face Transformers
        \item \textbf{开发工具}: Git、Docker、Linux/Unix环境、LaTeX、VS Code
        \item \textbf{其他}: SQL、MATLAB、基本的Web开发(HTML/CSS/JavaScript)
    \end{itemize}
    
    \item \textbf{语言能力}: 
    \begin{itemize}
        \item \textbf{中文}: 母语
        \item \textbf{英文}: 流利(CET-6 590分),具备学术阅读和写作能力
        \item \textbf{粤语}: 基础会话能力
    \end{itemize}
\end{itemize}

\vspace{-8pt}

% 荣誉奖项
\section*{荣誉奖项}

\begin{itemize}
    \item ICPC国际大学生程序设计竞赛区域赛银牌(重庆,2024)
    \item ICPC国际大学生程序设计竞赛区域赛银牌(济南,2024)
    \item 香港科技大学(广州)校级一等奖学金(2023-2024学年)
    \item 全国大学生数学建模竞赛省二等奖(2024)
\end{itemize}

\vspace{-8pt}

% 个人总结
\section*{个人总结}

香港科技大学(广州)人工智能专业本科生,具备扎实的计算机科学基础和算法设计能力。在ICPC竞赛中获得多项银牌,锻炼了高效的问题分析和解决能力。参与LLM相关科研项目,对人工智能前沿研究有实践经验。熟悉Python和C++开发,具备完整的项目实现经验。期待在研究生阶段深入研究人工智能安全、自然语言处理或多模态学习方向。

\end{document}