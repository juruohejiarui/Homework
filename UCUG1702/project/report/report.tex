\documentclass{article}

\usepackage[a4paper, total={6in, 9in}]{geometry}

\title{Molecular Dynamics Study of Water Using the TIP4P/2005 Model}
\author{Jiarui HE \\ UCUG1702}

\begin{document}
\maketitle

\section{Introduction}
The liquid water present at the air–water interface plays a crucial role in atmospheric chemistry,environmental processes, and a wide range of technologies.1–5 The structure of this interface governs phenomena such as aerosol reactivity, droplet stability, and wettability. In the bulk liquid phase, each water molecule generally forms approximately four hydrogen bonds, giving rise to a tetrahedral network. At the interface, however, this network is disrupted on the vapor side, which reduces the average number of hydrogen bonds and induces a preferential orientation of molecular dipoles and O–H bonds relative to the surface normal. A characteristic feature of this interfacial region is the presence of dangling hydrogens, which do not participate in hydrogen bonding and typically point toward the vapor phase.

Despite the apparent simplicity of the water molecule, liquid water exhibits remarkable properties at its free surface, including high surface tension and complex temperature-dependent changes in interfacial structure. These macroscopic properties are ultimately governed by temperature induced rearrangements of the hydrogenbond network and the orientational structure of interfacial molecules. In this work, we investigate (i) the distribution of hydrogen bonds formed by water molecules, (ii) the corresponding OH angles and dangling-hydrogen distributions relative to the surface normal and (iii) analyze of micro-structure form by water moleculars. By analyzing the temperature dependence of hydrogen bonding and OH orientational ordering at the interface, we aim to provide a microscopic understanding of the molecular structure of water at the airwater interface
\section{Methodology}

Molecular dynamics (MD) simulations with realistic water models can resolve interfacial structure at molecular length scales. Here, we use classical MD of a TIP4P/2005 water slab, a model that reproduces key thermodynamic and interfacial properties over a broad temperature range to examine how the hydrogen-bond network evolves at $T= 280,300,320 \rm{K}$

\bibliographystyle{nature}

\end{document}