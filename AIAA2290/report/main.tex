% CVPR 2025 Paper Template; see https://github.com/cvpr-org/author-kit

\documentclass[10pt,twocolumn,letterpaper]{article}

%%%%%%%%% PAPER TYPE  - PLEASE UPDATE FOR FINAL VERSION
\usepackage{cvpr}              % To produce the CAMERA-READY version
% \usepackage[review]{cvpr}      % To produce the REVIEW version
% \usepackage[pagenumbers]{cvpr} % To force page numbers, e.g. for an arXiv version

% Import additional packages in the preamble file, before hyperref
\input{preamble}

% It is strongly recommended to use hyperref, especially for the review version.
% hyperref with option pagebackref eases the reviewers' job.
% Please disable hyperref *only* if you encounter grave issues, 
% e.g. with the file validation for the camera-ready version.
%
% If you comment hyperref and then uncomment it, you should delete *.aux before re-running LaTeX.
% (Or just hit 'q' on the first LaTeX run, let it finish, and you should be clear).
\definecolor{cvprblue}{rgb}{0.21,0.49,0.74}
\usepackage[pagebackref,breaklinks,colorlinks,allcolors=cvprblue]{hyperref}

\usepackage{algorithm,algpseudocode,float}
\usepackage{lipsum}

\makeatletter
\newenvironment{breakablealgorithm}
  {% \begin{breakablealgorithm}
   \begin{center}
     \refstepcounter{algorithm}% New algorithm
     \hrule height.8pt depth0pt \kern2pt% \@fs@pre for \@fs@ruled
     \renewcommand{\caption}[2][\relax]{% Make a new \caption
       {\raggedright\textbf{\ALG@name~\thealgorithm} ##2\par}%
       \ifx\relax##1\relax % #1 is \relax
         \addcontentsline{loa}{algorithm}{\protect\numberline{\thealgorithm}##2}%
       \else % #1 is not \relax
         \addcontentsline{loa}{algorithm}{\protect\numberline{\thealgorithm}##1}%
       \fi
       \kern2pt\hrule\kern2pt
     }
  }{% \end{breakablealgorithm}
     \kern2pt\hrule\relax% \@fs@post for \@fs@ruled
   \end{center}
  }
\makeatother

%%%%%%%%% PAPER ID  - PLEASE UPDATE
\def\paperID{*****} % *** Enter the Paper ID here
\def\confName{CVPR}
\def\confYear{2025}

\title{Exploration of Adversarial Training}

%%%%%%%%% AUTHORS - PLEASE UPDATE
\author{Jiarui HE\\
50013538 \\
{\tt\small jhe218@connect.hkust-gz.edu.cn}
}

\begin{document}
\maketitle

\section{Introduction}

This paper first introduces adversarial training and other methods for defence again adversarial training. 
Then some evaluation and comparation of mentioned methods. The final part is the 


\section{Methods}
Let $\theta$ be the parameters of models, $f_\theta(\cdot)$ be the model, $\epsilon$ be the range of perturb of attack sample and $L$ be the loss function.

Then the target of adversarial training can be

$$
\theta^*=\arg\min_\theta\left\{
    \mathbb{E}_{(\mathbf{x}, \mathbf{y})\in \mathcal{D}}
    \left[
        \max_{\lVert\mathbf{\delta}\rVert\leq \epsilon}
            \left\{L(f_\theta(\mathbf{x}+\delta), y)\right\}
    \right]
\right\}
$$

There are some wide used algorithms for adversarial training.
For example, one of the most classical algorithms is the projected gradient descents (PGD).
Additionally, there are some algorithms which are base on this algorithms and make some modifies on it.
Furthermore, some structure applies adversarial training, such as GAN.

\subsection{Projected Gradient Descent}

This method try to construct attack samples using iteration towards the direction that relates to the gradient of loss function w.r.t. the input, 
which is usually the normalized value of $\nabla_{\mathbf{x}}(L(f_\theta(\mathrm{x}+\delta), y))$. People tend to use $l_0, l_2$ and $l_{\infty}$. Addtionally, the magnitude of iteration step is defined by a hyperparameter $\alpha$ .

The pesuado code of this algorithm is shown below:

\begin{breakablealgorithm}
\caption{PGD training}
\begin{algorithmic}[1] %每行显示行号
    \Require training set $\mathcal{D}$, learning rate $\tau$, training epochs $N$, hyperparameters $\alpha, \epsilon$, model $f_\theta$ and loss function $L$
    \State Initialize model parameters $\theta$

    \While 
    \EndWhile
\end{algorithmic}
\end{breakablealgorithm}


\end{document}
